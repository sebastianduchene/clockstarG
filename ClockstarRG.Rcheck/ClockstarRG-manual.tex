\nonstopmode{}
\documentclass[a4paper]{book}
\usepackage[times,inconsolata,hyper]{Rd}
\usepackage{makeidx}
\usepackage[utf8,latin1]{inputenc}
% \usepackage{graphicx} % @USE GRAPHICX@
\makeindex{}
\begin{document}
\chapter*{}
\begin{center}
{\textbf{\huge Package `ClockstarRG'}}
\par\bigskip{\large \today}
\end{center}
\begin{description}
\raggedright{}
\item[Type]\AsIs{Package}
\item[Title]\AsIs{ClockstaR-G: ClockstarR for genomic data sets}
\item[Version]\AsIs{0.1}
\item[Date]\AsIs{2014-12-25}
\item[Author]\AsIs{Sebastian Duchene}
\item[Maintainer]\AsIs{Sebastian Duchene }\email{sebastian.duchene@sydney.edu.au}\AsIs{}
\item[Description]\AsIs{ClockstaR-G is an implementation of ClockstarR that can be used for large genomic data sets. It also includes modification of the ClockstaR algorithm}
\item[Depends]\AsIs{R (>= 3.0.2), ape (>= 3.2), cluster(>= 1.15.3), doParallel (>=
1.0.8), foreach (>= 1.4.2)}
\item[License]\AsIs{GNU}
\end{description}
\Rdcontents{\R{} topics documented:}
\inputencoding{utf8}
\HeaderA{ClockstarRG-package}{ClockstaR-G: ClockstaR for genomic data sets}{ClockstarRG.Rdash.package}
\aliasA{ClockstarRG}{ClockstarRG-package}{ClockstarRG}
\keyword{ClockstaR, genome pacemakers}{ClockstarRG-package}
%
\begin{Description}\relax
ClockstaR-G is an implementation of ClockstaR for large data sets. It uses multicore packages to speed computation. It also includes a modification of the initial ClockstaR algorithm.
\end{Description}
%
\begin{Details}\relax

\Tabular{ll}{
Package: & ClockstarRG\\{}
Type: & Package\\{}
Version: & 0.1\\{}
Date: & 2014-12-25\\{}
License: & GNU\\{}
}
Please see examples in github
\end{Details}
%
\begin{Author}\relax
Sebastian Duchene
Maintainer: Sebastian Duchene <sebastian.duchene@sydney.edu.au>
\end{Author}
%
\begin{References}\relax
Please see original paper. 
\end{References}
%
\begin{SeeAlso}\relax
ClockstaR
\end{SeeAlso}
\inputencoding{utf8}
\HeaderA{get\_scaled\_brs}{get\_scaled\_brs: Obtain a matrix with scaled branch lengths for a list of trees of class multiphylo.}{get.Rul.scaled.Rul.brs}
%
\begin{Description}\relax
This function scales the branch lengths of trees and returns a matirx with the names of trees in the columns and the scaled branch lengths in the rows.
\end{Description}
%
\begin{Usage}
\begin{verbatim}
get_scaled_brs(tree_list)
\end{verbatim}
\end{Usage}
%
\begin{Arguments}
\begin{ldescription}
\item[\code{tree\_list}] 
List of trees of class phylo. This can be of class multiphylo. The trees should have the same number of taxa, and identical topologies.

\end{ldescription}
\end{Arguments}
%
\begin{Value}
Matrix with trees along the columbs and scaled branch legths along the rows.
\end{Value}
%
\begin{Author}\relax
Sebastian Duchene
\end{Author}
\inputencoding{utf8}
\HeaderA{optim\_clusters\_coord}{Select optimal number of clusters using the Gap statistic}{optim.Rul.clusters.Rul.coord}
%
\begin{Description}\relax
optim\_clusters\_coord obtains the optimal number of clusters, the cluster assignment, and other cluster statistics for the optimal number of clusters, obtained with the Gap statistic. 
\end{Description}
%
\begin{Usage}
\begin{verbatim}
optim_clusters_coord(coord_mat, n_clusters = 2, kmax, b_reps = 100, out_cluster_id = "opt_clus_id.txt", out_cluster_info = "opt_clusinfo_sbsd.txt", out_gap_stats = "gap_stats.txt", plot_clustering = F)
\end{verbatim}
\end{Usage}
%
\begin{Arguments}
\begin{ldescription}
\item[\code{coord\_mat}] 
This is the matrix of the scaled branch lengths of the gene trees. It can be obtained with get\_scaled\_brs.

\item[\code{n\_clusters}] 
Number of computing clusters. This is the number of CPUs to use. Do not confuse with the number of clocks or pacemakers.

\item[\code{kmax}] 
Maximum number of clusters to test. This is k for 1<k<N

\item[\code{b\_reps}] 
Bootstrap replicates

\item[\code{out\_cluster\_id}] 
Name of the file to save the cluster assignment for each gene. 

\item[\code{out\_cluster\_info}] 
Name of the file to save the cluster information for each gene. This includes issolation, mean dissimilarity, and cluster size.

\item[\code{out\_gap\_stats}] 
Name of file to save the Gap statistic for values of k.

\item[\code{plot\_clustering}] 
boolean. Should ClockstaR-G plot the Gap statistics.

\end{ldescription}
\end{Arguments}
%
\begin{Value}
\begin{ldescription}
\item[\code{optimal\_k }] The optimal value for k.
\item[\code{cluster\_info }] matrix with information for each of the clusters (referred to as pacemakers in the publication)
\item[\code{cluster\_id }] matrix with the clustering assignment for each gene
\item[\code{gap\_statistics }] matrix with the Gap statistics
\item[\code{alt\_gap }] difference between sucessive mean Gap statistics

\end{ldescription}
\end{Value}
%
\begin{Author}\relax
Sebastian Duchene
\end{Author}
%
\begin{References}\relax
Pending.
\end{References}
\inputencoding{utf8}
\HeaderA{scaled\_brs}{Scale branch lengths of tree.}{scaled.Rul.brs}
%
\begin{Description}\relax
This function is for internal usage. It scales the branch lengths of the trees.
\end{Description}
%
\begin{Usage}
\begin{verbatim}
scaled_brs(tree)
\end{verbatim}
\end{Usage}
%
\begin{Arguments}
\begin{ldescription}
\item[\code{tree}] 
tree with branch lengths. It should be phylo format.

\end{ldescription}
\end{Arguments}
%
\begin{Value}
A vector of scaled branch lengths.

Sebastian Duchene


\end{Value}
\printindex{}
\end{document}
